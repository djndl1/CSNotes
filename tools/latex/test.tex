\documentclass[a4paper, 11pt]{article}

% Macros
\newcommand{\TUG}{Tex Users Group\xspace}
\newcommand{\keyword}[1]{\textbf{#1}}

% If a dot, a comma, an exclamation or a quotation mark follows, it won't insert a space, but if a normal letter 
% follows, then it will.
\usepackage{xspace}
\usepackage{minted}

\begin{document}

Text can be \emph{emphasized}.

Besides being \textit{italic} words could be \textbf{bold},
\textsl{slanted} or typeset in \textsc{Small Caps}.

Such commands can be \textit{\textbf{nested}}.


\emph{See how \emph{emphasizing} looks when nested.}
\texttt{http://www.ctan.rog}

\textrm{Roman text}

\section{\sffamily\LaTeX resources in the internet}    % sans-serif font declaration
\rmfamily The best place for downloading LaTex related software is CTAN 
Its address is \texttt{http://www.ctan.org \sffamily(CTAN)} 

% We use curly braces to tell LaTex where to apply a command and where to stop that
% an opening curly brace tells LaTex to begin a group
{\sffamily
Text can be {\em emphasized}.

Besides being {\itshape italic} words could be {\bfseries bold}, {\slshape slanted} or typeset in {\scshape Small Caps}.

Such commands can be {\itshape \bfseries nested.} 

{\em See how {\em emphasizing looks when nested}}
}
 
\title{\sffamily Sans serif}
\maketitle
Hi

% font size

\noindent \tiny tiny \scriptsize scriptsize \footnotesize footnotesize \small small, \normalsize normalsize \large large \Large Large \LARGE LARGE 
\huge huge and \Huge Huge \bigskip

\normalsize  The actual resulting font size depends on the base font. One way to apply font size is to use environment:

\begin{huge}
A huge example
\end{huge}
\bigskip % skip some space vertically
This is just another small illustrative example.

\LaTeX allows you to define our own commands in the preamble, called \emph{macros}.

The \TUG is an organization for people who are interested in \TeX\ or \LaTeX.

\keyword{Grouping} by curly braces limits the \keyword{scope} of \keyword{declarations}.

\mint{latex}|\newcommand{command}[arguments][optional]{definition}|

where arguments are integers from 1 to 9. It is the key to introduce logical formatting. We should avoid using \LaTeX font commands inside the document.

\end{document}

